\documentclass[8pt,a4paper,landscape,oneside]{amsart}
\usepackage{amsmath, amsthm, amssymb, amsfonts}
\usepackage[T1]{fontenc}
\usepackage[utf8]{inputenc}
\usepackage{booktabs}
\usepackage{caption}
\usepackage{color}
\usepackage{fancyhdr}
\usepackage{float}
\usepackage{fullpage}
\usepackage{subcaption}
\usepackage[scaled]{beramono}
\usepackage{titling}
\usepackage{datetime}
\usepackage{enumitem}
\usepackage{multicol}
\usepackage{bm}
\usepackage{dirtytalk}

% Minted
\usepackage{minted}
\newcommand{\codej}[1]{\inputminted[fontsize=\large,tabsize=2,baselinestretch=1]{java}{code/#1}}
\newcommand{\codec}[1]{\inputminted[fontsize=\large,tabsize=2,baselinestretch=1]{cpp}{code/#1}}
\newcommand{\codep}[1]{\inputminted[fontsize=\large,tabsize=2,baselinestretch=1]{py}{code/#1}}
\newcommand{\codeb}[1]{\inputminted[fontsize=\large,tabsize=2,baselinestretch=1]{bash}{code/#1}}
\newcommand{\codev}[1]{\inputminted[fontsize=\large,tabsize=2,baselinestretch=1]{vim}{code/#1}}


\newcommand{\subtitle}[1]{%
  \posttitle{%
    \par\end{center}
    \begin{center}\large#1\end{center}
    \vskip0.1em\vspace{-1em}}%
}
\setlength{\headheight}{15.2pt}
\setlength{\droptitle}{-20pt}

\posttitle{\par\end{center}}
\renewcommand{\headrulewidth}{0.4pt}
\renewcommand{\footrulewidth}{0.4pt}

\usepackage{fancyhdr}
\pagestyle{fancy}
\fancyhf{}
\rhead{Page \thepage}
\lhead{from \_\_future\_\_ import solution -- Lunds Universitet}


\newcommand{\bigO}{\mathcal{O}}

\title{from \_\_future\_\_ import solution}
\subtitle{Team Reference Document}
\author{}
\date{\ddmmyyyydate{\today{}}}

\begin{document}

\begin{multicols*}{2}
\maketitle
\thispagestyle{fancy}
\vspace{-3em}

\tableofcontents
\begin{large}
\section{Achieving AC on solved problems}
\subsection{WA}
\begin{itemize}
    \item Does test input pass?
    \item Is overflow possible?
    \item Is it possible that we solved the wrong problem - reread the problem statement!
    \item Start creating small test cases.
    \item Does cout print with sufficient percision?
    \item Abstract the implementation!
\end{itemize}
\subsection{TLE}
\begin{itemize}
    \item Is the solution sanity checked? (Time complexitity)
\end{itemize}
\subsection{RTE}
\subsection{MLE}
\section{Standard easy methods}
\subsection{BFS-DFS}
\section{Templates, etc.}
\subsection{C++}
\codec{template.cpp}
\codec{input.cpp}
\subsection{Python}
If a specified output, say $o$, of decimals, say $x$, should be given then write $$print '\%.xf' \% o$$
\section{Data Structures}
\subsection{Fenwick Tree}
\codep{fenwicktree.py}
\codej{fenwicktree.java}
\subsection{Segment Tree}
\codep{sgmtree.cpp}
\subsection{Lazy Segment Tree}
\codec{lazysgmtree.cpp}
\subsection{Union Find}
\codep{unionfind.py}
\subsection{Monotone Queue}
\subsection{Treap}
\section{Graph Algorithms}
\subsection{Distance from source to all nodes (pos weights) - Djikstra's algorithm}
\codep{djikstra.py}
\subsection{Distance from source to all nodes (neg weights) - Bellman Ford}
\codep{bellmanford.py}
\subsection{All distances in graph (neg weights) - Floyd Warshall}
\codep{floydwarshall.py}
\subsection{Bipartite graphs}
\codec{kuhns.cpp}
\subsection{Network flow}
\codec{dinic.cpp}
\codep{maxflow.py}
\codep{maxflow2.py}
\subsection{Min cost max flow}
\codep{mincostmaxflow.py}
\codej{mincostmaxflow.java}
\section{Dynamic Programming}
\subsection{Longest increasing subsequence}
\codep{lis.py}
\subsection{String functions}
\codep{stringmatching.py}
\subsection{Josephus problem}
\codep{josephus.py}
\subsection{Knapsack}
\codep{knapsack.py}
\section{Coordinate Geometry}
\subsection{Area of polygon}\label{sec: polyarea}
\codep{polygonArea.py}
\subsection{Distance between line, segment and point}
Note that errors have been found in the code, and the code has not been checked thoroughly.
\codep{linesegdist.py}
\subsection{Pick's theorem}
Pick's theorem states that the area, $A$, of a polygon with lattice coordinates for its corners is given by $$A=I+\frac{B}{2}-1,$$ where $B$ is the number of boundary lattice points and $I$ is the number of interior lattice points. This can often be used to find the number of interior points of a polygon since the area is easily computed, see \ref{sec: polyarea}, and the number of boundary lattice points is calculated as follows:
\codep{boundarypoints.py}
\subsection{Implementations}
\section{Math}
\subsection{System of equations}
\codep{gaussianelimination.py}
\subsection{Number Theory}
\codep{gcdbezout.py}
\subsection{Primes and Prime factorization}
\codep{primecalc.py}
\subsection{Chinese remainder theorem}
\codep{crt.py}
\section{Other things}
\subsection{Convex Hull}
\codep{convexhull.py}
\subsection{Fast Fourier Transform}
\codec{fft.cpp}
\subsection{Large Primes}
\begin{itemize}
    \item 133469857
    \item 1519262429
    \item 17024073439
    \item $10^9+7$
    \item $10^9+9$
    \item Add prime on the form $t\cdot 2^x+1$ to use for FFT.
\end{itemize}
\section{Practice Contest Checklist}
\begin{itemize}
    \item Operations per second in py2
    \item Operations per second in py3
    \item Operations per second in java
    \item Operations per second in c++
    \item Operations per second on local machine
    \item Is MLE called MLE or RTE?
    \item What happens if extra output is added? What about one extra new line or space?
    \item Look at documentation on judge.
    \item Submit a clar.
    \item Print a file.
    \item Directory with test cases.
\end{itemize}


\end{large}
\end{multicols*}
\end{document}
