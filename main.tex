\documentclass[8pt,a4paper,landscape,oneside]{amsart}
\usepackage{amsmath, amsthm, amssymb, amsfonts}
\usepackage[T1]{fontenc}
\usepackage[utf8]{inputenc}
\usepackage{booktabs}
\usepackage{caption}
\usepackage{color}
\usepackage{fancyhdr}
\usepackage{float}
\usepackage{fullpage}
\usepackage{subcaption}
\usepackage[scaled]{beramono}
\usepackage{titling}
\usepackage{datetime}
\usepackage{enumitem}
\usepackage{multicol}
\usepackage{bm}
\usepackage{dirtytalk}

\setcounter{secnumdepth}{4}
% Minted
\usepackage{minted}
\newcommand{\codej}[1]{\inputminted[fontsize=\large,tabsize=2,baselinestretch=1]{java}{code/#1}}
\newcommand{\codec}[1]{\inputminted[fontsize=\large,tabsize=2,baselinestretch=1]{cpp}{code/#1}}
\newcommand{\codep}[1]{\inputminted[fontsize=\large,tabsize=2,baselinestretch=1]{py}{code/#1}}
\newcommand{\codeb}[1]{\inputminted[fontsize=\large,tabsize=2,baselinestretch=1]{bash}{code/#1}}
\newcommand{\codev}[1]{\inputminted[fontsize=\large,tabsize=2,baselinestretch=1]{vim}{code/#1}}


\newcommand{\subtitle}[1]{%
  \posttitle{%
    \par\end{center}
    \begin{center}\large#1\end{center}
    \vskip0.1em\vspace{-1em}}%
}
\setlength{\headheight}{15.2pt}
\setlength{\droptitle}{-20pt}

\posttitle{\par\end{center}}
\renewcommand{\headrulewidth}{0.4pt}
\renewcommand{\footrulewidth}{0.4pt}

\usepackage{fancyhdr}
\pagestyle{fancy}
\fancyhf{}
\rhead{Page \thepage}
\lhead{from \_\_future\_\_ import solution -- Lunds Universitet}


\newcommand{\bigO}{\mathcal{O}}

\title{from \_\_future\_\_ import solution}
\subtitle{Team Reference Document}
\author{}
\date{\ddmmyyyydate{\today{}}}

\begin{document}

\begin{multicols*}{2}
\maketitle
\thispagestyle{fancy}
\vspace{-3em}

\tableofcontents
\begin{large}
\section{Achieving AC on solved problems}
\subsection{WA}
\begin{itemize}
    \item Edge cases (minimal input, etc)?
    \item Does test input pass?
    \item Is overflow possible?
    \item Is it possible that we solved the wrong problem? Reread the problem statement!
    \item Start creating small test cases.
    \item Does cout print with sufficient percision?
    \item Abstract the implementation!
\end{itemize}
\subsection{TLE}
\begin{itemize}
    \item Is the solution sanity checked? (Time complexitity)
    \item Use pypy / rewrite in C++ and Java
    \item Can DP be applied in some part?
    \item Try creating worst case input to see how far away the solution is
    \item Binary search instead of exhaustive search
    \item Binary search over the answer
\end{itemize}
\subsection{RTE}
\begin{itemize}
    \item Recursion limit in python?
    \item Out of bounds?
    \item Division by 0?
    \item Modifying something we iterate over?
    \item Not using well defined sorting?
    \item If nothing makes sense, try binary search with try-catch/except.
\end{itemize}
\subsection{MLE}
\begin{itemize}
    \item Create objects outside recursive function
    \item Rewrite recursive solution to iterative
\end{itemize}
\section{Templates, etc.}
\subsection{C++}
\codec{template.cpp}
\codec{input.cpp}
\subsection{Python}
\codep{template.py}
\section{Data Structures}
\subsection{Fenwick Tree}
\codep{fenwicktree.py}
\codec{fenwicktree.cpp}
\subsection{Segment Tree}
\codep{sgmtree.cpp}
\subsection{Lazy Setting Segment Tree}
\codec{lazysetsgmtree.cpp}
\subsection{Lazy Incrementing Segment Tree}
\codec{lazyincsgmtree.cpp}
\subsection{Union Find}
\codep{unionfind.py}
\subsection{Monotone Queue}
\codep{monotonequeue.py}
\subsection{Treap}
\codec{Treap.cpp}
\section{Graph Algorithms}
\subsection{Distance from source to all nodes (pos weights) - Djikstra's algorithm}
\codep{djikstra.py}
\subsection{Distance from source to all nodes (neg weights) - Bellman Ford}
\codep{bellmanford.py}
\codec{bellmanford.cpp}
\subsection{All distances in graph (neg weights) - Floyd Warshall}
\codep{floydwarshall.py}
\subsection{Bipartite graphs}
\codec{kuhns.cpp}
\subsection{Network flow}
\codec{dinic.cpp}
\codep{maxflow.py}
\codep{maxflow2.py}
\subsection{Min cost max flow}
\codep{mincostmaxflow.py}
\codej{mincostmaxflow.cpp}
\subsection{Topological sorting - for example finding DAG order}
\codep{topsort.py}
\subsection{2sat}
\codec{2sat.cpp}
\section{Dynamic Programming}
\subsection{Longest increasing subsequence}
\codep{lis.py}
\subsection{String functions}
\codep{stringmatching.py}
\subsection{Josephus problem}
\codep{josephus.py}
\subsection{Knapsack}
\codep{knapsack.py}
\section{Coordinate Geometry}
\subsection{Area of polygon}\label{sec: polyarea}
\codep{polygonArea.py}
\subsection{General geometry operations on lines, segments and points}
\codep{geometry.py}
\subsection{Pick's theorem}
Pick's theorem states that the area, $A$, of a polygon with lattice coordinates for its corners is given by $$A=I+\frac{B}{2}-1,$$ where $B$ is the number of boundary lattice points and $I$ is the number of interior lattice points. This can often be used to find the number of interior points of a polygon since the area is easily computed, see \ref{sec: polyarea}, and the number of boundary lattice points is calculated as follows:
\codep{boundarypoints.py}
\subsection{Convex Hull}
\codep{convexhull.py}
\section{Math}
\subsection{System of equations}
\codep{gaussianelimination.py}
\subsection{Number Theory}
\codep{gcdbezout.py}
\subsection{Primes and Prime factorization}
\codep{primecalc.py}
\subsection{Chinese remainder theorem}
\codep{crt.py}
\subsection{Finding primitive root}
\codep{primitiveroot.py}
\subsection{Baby-step-giant-step algorithm}
\codep{babystepgiantstep.py}
\section{Other things}
\subsection{Fast Fourier Transform}
\codec{fft.cpp}
\subsection{Large Primes}
\begin{itemize}
    \item 133469857
    \item 1519262429
    \item 17024073439
    \item 3435975962563
    \item 22732918586849
    \item 22734054029887
    \item $10^9+7$
    \item $10^9+9$
    \item $13631489 = 2^{20}\cdot 13 + 1$
    \item $120586241 = 2^{20}\cdot 5\cdot 23 + 1$
    \item $998244353 = 2^{23}\cdot 7\cdot 17 + 1$
\end{itemize}
\subsection{Scheduling}
\codec{scheduling.cpp}
\section{Practice Contest Checklist}
\begin{itemize}
    \item Operations per second in py2
    \item Operations per second in py3
    \item Operations per second in java
    \item Operations per second in c++
    \item Operations per second on local machine
    \item Is MLE called MLE or RTE?
    \item What happens if extra output is added? What about one extra new line or space?
    \item Look at documentation on judge.
    \item Submit a clar.
    \item Print a file.
    \item Directory with test cases.
\end{itemize}
\newpage
\renewcommand{\labelitemi}{\textendash}
\section{Methods and ideas}
\noindent
Use some characteristics of the problem (i.e. bounds)
\begin{itemize}
  \item $N\leq10$: Exhaustive search $N!$
  \item $N\leq20$: Exponential
  \item $N\leq 10^4$: Quadratic
  \item $N\leq 10^6$: Has to be NlogN
\end{itemize}
Greedy
\begin{itemize}
  \item Invariants
  \item Scheduling
\end{itemize}
BFS/DFS\\
DP\\
Binary search
\begin{itemize}
  \item Over the answer
  \item To find something in sorted structure
\end{itemize}
Flow
\begin{itemize}
  \item Min-cost-max-flow
  \item Run the flow and look at min cut
  \item Regular flow
  \item Matching
\end{itemize}
View the problem as a graph\\
Color the graph\\
When there is an obvious TLE solution
\begin{itemize}
  \item Use some sorted data structure
  \item In DP, drop one parameter and recover from others
  \item Is something bounded by the statement?
  \item In DP, use FFT to reduce one N to logN
\end{itemize}
Divide and conquer - find interesting points in NlogN\\
Square-root tricks
\begin{itemize}
  \item Periodic rebuilding: every $\sqrt{n}$, rebuild static structure.
  \item Range queries: split array into segments, store something for each segment.
  \item Small and large: do something for small(with low degree) nodes and something else for large nodes.
  \item If the sum of some parameters is small, then the number of different sized parameters is bounded by roughly $\sqrt{n}$.
\end{itemize}
Hall's marriage theorem\\
Combinatorics / Number theory / Maths
\begin{itemize}
  \item Inclusion/Exclusion
  \item Fermat's little theorem / Euler's theorem
  \item NIM
\end{itemize}
Randomization
\begin{itemize}
  \item Finding if 3 points are on the same line
  \item Checking matrix equality by randomizing vector and multiply
\end{itemize}
Geometry
\begin{itemize}
  \item Cross product - to check order of points / area
  \item Scalar product
\end{itemize}

\end{large}
\end{multicols*}
\end{document}
